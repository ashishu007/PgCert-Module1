\section{Identified Impact}\label{sec:impact}
This research will introduce new state-of-the-art methodologies for solving the NLP related problems in business processes. Machine learning research has evolved rapidly with the introduction of deep learning techniques build upon the support of tremendous computational power. But still there use in business processes is very few due to the difference between nature of problems in real-world applications and academic research. We will bring attention to the academic research community towards the problems faced by business processes that are not well addressed right now. Our work will help industries make their workforce allocation better as many tasks which do not need much human intervention will be handled in an automated manner. This will free up the human resource of a company for more creative and innovate tasks. Eventually leading in the cost reduction as they have to focus on quality of their workforce instead of quantity.

Right now the work done by few industries to automate their business process is not made available in public domain. There are a few conglomerates who work on developing methods to address some of their problems but they tend not to open-source their findings in order to gain financial profits. The findings from our project will be made public through various publications and code sharing website like GitHub \footnote{\url{https://github.com/}} that can help industries with lesser resources to enhance their performance. Instead of using their resources on automating their text related processes they can focus on their core issues \& ideas whereas our work can help them in automation.

% We plan to build upon current
% deep learning techniques on HAR, adapt and improve them for multi-modal sensor reasoning for exercises.
% With objective 1 we will bring together different machine learning data domains such as time
% series, images and video as we create a multi-modal spatio-temporal dataset. It will be adaptable to
% different application domains as well as future research in HAI. It will also contribute towards evaluating
% transfer learning capabilities of machine learning models. Objective 2 will introduce new attention
% mechanisms for sensor data selection in abstract levels and in objective 3 we will introduce strategies
% to improve robustness in deep learning architectures.Objective 4 will implement novel deep learning
% models for qualitative evaluation which involves similarity comparison of spatio-temporal data.
% From the healthcare application perspective this research will contribute towards a sustainable
% digital intervention for MSD prevention and self-management while raising awareness. MSD has
% directly affects the workforce of a country with sedentary lifestyles and as a result it has a major
% impact on economical and social status of the country. This digital intervention will enable the user
% to perform physiotherapist recommended exercises at home with supervision from the qualitative
% evaluation component. We plan to involve users in each step of this research and raise awareness. We
% will seek user feedback on the utility of the digital intervention to support, maintain and encourage
% an active lifestyle in the prevention and management of MSDs

% The major outcome expected of this project is the development and production
% of a sustainable profile generation and long term health condition risk determination system, designed to scale with additional FitHomes installations as they
% are constructed. As new residents move into homes, labelled examples will be
% produced with a slight involvement on their end, ensuring personalised profiles
% are produced based on their regular performances of activities. Bespoke and
% retrofit sensor environments will also be supported. This will ensure that the
% FitHomes project can safely expand and continue to provide state-of-the-art
% sensor analytics to residents.
% It was proposed that 2-3 impactful publications may be produced as a result
% of this project. In the long term, we can expect the FitHomes project as a whole
% will provide a very useful unique dataset for data mining. This may potentially
% allow for future research on seemingly unrelated and unknown risk factors for a
% range of long term health conditions to be highlighted. Other long term studies
% in home environments may also be able to make use of any publicly distributed
% FitHomes datasets which would have a high impact in the field. Multi-modal
% sensor analytics as a field could benefit from this project.


% This will help several industries to automate their process in various domains
