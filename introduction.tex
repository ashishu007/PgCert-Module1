\section{Introduction}
The evolution of the Web combined with the emergence of cloud computing has led to an enormous growth in the amounts of data and information available for organisations to employ in their business processes. The statistics suggest 80\% of business-relevant information originates in unstructured form, typically text; the volume of generated data is growing with predictions that by 2023, we’ll be sending over 350 billion emails daily \cite{email_site}. These statistics make a strong case for the development of effective approaches for natural language processing. 

Sentiment analysis is on area of natural language processing that has delivered good performance, largely due to the general nature of the challenge \cite{sun2019fine,yang2019xlnet}. Effective solutions have been developed, initially employing lexicon-based approaches \cite{mukras2007selecting}, but more recently supplemented with Machine Learning and Deep Learning approaches \cite{devlin2018bert}. Solutions have become more refined in looking at aspect-based sentiment analysis \cite{bandhakavi2018context} and more fine grained in looking at emotion analysis \cite{bandhakavi2017lexicon}. As a result of the improved performance from these developments, sentiment analysis is now a commonplace approach employed by industry. 

The challenges is to develop approaches that go beyond sentiment analysis and to develop NLP solutions supporting business processes more generally \cite{budek_2019}. One on-going requirement is to support the extraction of actionable knowledge from text for custom business processes \cite{Bordignon2018bpm}. Custom rule-based solutions have been developed that typically fill out slots in predefined templates. But these methods lack in giving good performance as well as they are too domain specific. By just changing the problem domain, similar kind of problems can not be solved with the same approach. In this project we hope to develop machine and deep learning based approaches that can outperform the custom rule-based solutions and make similar performance gains that has been seen recently in sentiment analysis. 

There are many business processes that require NLP solutions like document classification, text generation/summarisation, or knowledge extraction as part of the operation workflow e.g. risk assessment, compliance management, auditing, recruitment, and procurement \cite{Sintoris2017bpm}. In many industries, compliance management involve use of a large number of regulatory documents from different statutory bodies at various scenarios. The possible problems statements can be: sorting and prioritizing the documents based on their source organisation, e.g., HSE, BSI; each regulatory document may contain a large number of compliance requirement (CR) that needs to be identified and extracted; after extraction the CR needs to classified by importance and assigned to different job roles. 

There are multiple challenges in relation to these problem statements;

% In this project the intention is to give a clear focus by initially concentrating on one problem. In the recent OGIC funded project at CSDM School, developing a compliance management system has identified a suitable use case. It has become apparent that the state-of-the-art is not yet sufficiently developed to provide an effective practical solution for the extraction of compliance requirements from regulatory documents.


\begin{itemize}
    \item NLP techniques can struggle when they encounter proprietary formats, such as pdf, because the structure of the document on which approaches may rely is lost.
    \item Automated business processes are expected to perform with extremely high accuracy in order to eliminate the manual intervention. Employing feedback loops is one approach to improving accuracy but are expensive to embed in rule-based techniques.
    \item Business processes may have unbalanced cost of error. For example, in case of CR extraction the focus is on minimising the false positives.
\end{itemize}

There are alternative use cases, e.g. assessing applicants’ suitability for a job from analysis of CVs. For a company getting hundreds or thousands of applications for some job roles, it would very difficult for them to hire a lot of experts in order to scrutinize the CVs. It would be very helpful for such company if an automated system is developed capable of filtering these CVs and selecting suitable candidates for further processes. The alternative use cases share similar kind of challenges.

Machine and Deep learning approaches can address the above challenges but rely on availability of labelled data, which is expensive to acquire. Business processes get labelled data as a part of their process. It is not feasible to wait for a long amount of time to get enough labelled data for supervised training. We need to build systems which are reliable with less training data as well. This scenario is referred as cold start problem where we have very minimal labelled to start with and as time goes on the labelled data builds up gradually as well.

The rest of the paper is organised as follows: in chapter \ref{sec:rq}, we talk about the research question and the objectives we hope to meet in order to answer the question; in chapter \ref{sec:lit_rev}, we review the literature and talk about different tasks and methods in natural language processing; in the next chapter \ref{sec:methods}, we present our methodology that we hope to adopt in order to meet our objectives followed by some initial work in chapter \ref{sec:initial_work} and identified impact in chapter \ref{sec:impact}. Finally in chapter \ref{sec:time_plan}, we conclude the paper with time management plan for next three years of the PhD.