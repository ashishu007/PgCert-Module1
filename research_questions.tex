\section{Research Questions and Objectives}\label{sec:rq}
This research will explore the problems involved in text mining in a business process with the goal of improving performance of current state-of-the-art performance. The primary research question for this project will be:
% a goal of developing better system capable of increasing the current performance.
% aim towards developing the a system capable of extracting knowledge from text and mapping that information to a specific business requirement.

\vspace{12pt}

\noindent \textbf{Can Machine Learning help improve the performance of natural language processing related problems in business processes?}

\vspace{12pt}
% \noindent 
Business processes use different automated systems to solve the NLP related tasks. They mostly use the state-of-the-art methods proposed by academic research which is not always enough as the environment where academic research is conducted is different from the real-world scenario of a business process. We hope to meet following objectives in order to answer the research question.

\paragraph{Objectives}\label{sec:objectives}

\begin{itemize}

    \item[\textbf{O1}] Identifying the NLP related problems in business processes that are not well addressed by academic research.
    % research gaps in literature when applying text mining to business processes.
    % Identifying the problems in business processes with respect to text mining that are not addressed in general academic research.
    \begin{itemize}
        % \item[-] What are the problems that arise while implementing literature methods on real-world data in real-time scenario?
        \item[-] Academic research simplifies a real-world problem in order to gain good performance like working with a pre-processed labelled data.
        \item[-] But in a real world scenario, enough labelled data might not be available for training of learning algorithms. The data is generated as a part of the business process.
    \end{itemize}
    
    % in Cold Start Scenario.
    \item[\textbf{O2}] Identifying the common use-cases from business processes and gathering data for those use-cases.
    \begin{itemize}
        % \item[-] What are the most common use cases in business processes in terms of text mining problems.
        \item[-] There are different types of problems related to NLP in business process like natural language generation, text classification, or question answering.
        % \item[-] We 
        \item[-] We will identify some use-cases that are very common throughout the industry.
        \item[-] While selecting the use-case, focus will also be on the availability of datasets.
    \end{itemize}
    
    % Selecting datasets and identifying the use case.
    \item[\textbf{O3}] Setting up the benchmark performance by implementing state-of-the-art algorithms for the real-world scenario.
    \begin{itemize}
        % \item[-] How can we make use of public datasets to simulate a business process scenario in order to test current state-of-the-arts on real world problems.
        \item[-] By using public datasets, we will simulate the real-world scenario to apply the current state-of-the-art algorithms. For example, this can be achieved by applying time stamps on the public datasets. 
        \item[-] This will identify the performance of current state-of-the-art algorithms on real-world scenario.
        % For example, adding timestamps to public datasets 
    \end{itemize}
    
    \item[\textbf{O4}] Developing novel methods to outperform the current state-of-the-art algorithms and evaluating them with real world data on different use-cases.
    \begin{itemize}
        \item[-] After we identify the gaps between business processes and academic research we will propose our algorithms that can enhance the performance of NLP systems in business processes.
        \item[-] We will also evaluate our methods on real-world scenario, if possible, with some industrial data.
    \end{itemize}
    % Proposing novel methods on tackling the problems.
    
    % \item Evaluating the proposed methods in real-world cold start scenario in different use cases.
\end{itemize}


% \begin{itemize}
%     \item [\textbf{O1}] Identify the gap between problems in business process and academic research in terms of natural language processing.
%     \begin{itemize}
%         \item There is a huge difference in the types of problems faced in business process as compared to academic research.
%         \item For example, in a business process the labelled data is generated in streams as a part of the process. Whereas in academic research, we develop solutions for the pre-processed  quality dataset.
%         \item We will identify these kinds of gaps and propose methods to solve them.
%     \end{itemize}
    
%     % \item [\textbf{O2}] Propose the solution for those identified gaps.
%     % \item [O1] Develop novel deep learning algorithms that can improve the performance of text mining in business process. 
    
%     \item [\textbf{O2}] Develop novel techniques to handle the textual data in cold start scenario for business process.
%     \begin{itemize}
%         \item Cold start is one of the gaps between business process and academia.
%         \item Either we have small amount of labelled data or huge amount of unlabelled data to start with.
%         \item We will develop novel techniques to handle this scenario.
%     \end{itemize}
%     % \item [O3] Develop the mechanism for retraining the model on original data instead of synthetic data after an interval.
% \end{itemize}
